% Author:Zhuming Shi, Peking University
% Theme from https://github.com/matze/mtheme

% \documentclass[12pt,AutoFakeBold,aspectratio=169,mathserif]{beamer}
\documentclass[AutoFakeBold]{beamer}
\usepackage[english]{babel}

\usetheme{metropolis}

\usepackage{fontspec}% 控制字体
\setmainfont{Times New Roman}% 英文字体
\newfontfamily\arial{Arial}% Arial字体

\usepackage{xeCJK} % 中文支持
\setCJKmainfont{SimHei} % 中文字体
\XeTeXlinebreaklocale "zh"%中文自动换行
\XeTeXlinebreakskip = 0pt plus 1pt%中文自动换行

\usepackage{graphicx}
\usepackage{subfigure}
\usepackage{caption}

\usepackage{amsthm,amsmath,amssymb,mathrsfs}% 数学符号和花体支持
\usepackage{booktabs}% 绘制三线表
\usepackage{latexsym}% 绘制特殊数学符号
\usepackage{siunitx}% 数学模式中使用SI单位

\usepackage[version=3]{mhchem}% 化学反应式
\usepackage{epstopdf}% 插入ChemDraw的.eps结构图

% 代码环境
\usepackage{listings}
\usepackage{color}

\definecolor{dkgreen}{rgb}{0,0.6,0}
\definecolor{gray}{rgb}{0.5,0.5,0.5}
\definecolor{mauve}{rgb}{0.58,0,0.82}

\lstset{frame=tb,
  language=c++,
  aboveskip=3mm,
  belowskip=3mm,
  showstringspaces=false,
  columns=flexible,
  basicstyle={\small\ttfamily},
  numbers=none,
  numberstyle=\tiny\color{gray},
  keywordstyle=\color{blue},
  commentstyle=\color{dkgreen},
  stringstyle=\color{mauve},
  breaklines=true,
  breakatwhitespace=true,
  tabsize=3
}

\setbeamerfont{footnote}{size=\tiny}

\newcommand{\unknow}[1]{{\arial \textbf{#1}}}%未知化合物格式
\newcommand{\substance}[1]{\textbf{\emph{#1}}}%矿物名称格式

% \setbeamertemplate{background}{\includegraphics[height=\paperheight]{figures/sisters1080.png}}

\makeatletter 
\renewcommand{\@thesubfigure}{\hskip\subfiglabelskip}
\makeatother

\title{网络编程3}
\author{施朱鸣}
\date{12月24日}

\begin{document}
    {
    % \setbeamertemplate{background}{\includegraphics[height=\paperheight]{figures/sisters1080.png}}
    % \setbeamertemplate{background}{\includegraphics[height=\paperheight]{figures/misaka1080.jpg}}
    \setbeamertemplate{background}{\includegraphics[height=\paperheight]{figures/misaka43.jpg}}
    
    \begin{frame}
    \titlepage
    \end{frame}
    \frame{\frametitle{Outline}\tableofcontents[hideallsubsections]}

    \section{网络请求}
    \begin{frame}
        \frametitle{怎么进行HTTP传输?}
    
        \begin{itemize}
            \item 超文本传输协议\begin{itemize}
                \item 超在哪里?可以包含超链接<a herf="..."> </a>
            \end{itemize}
            \item 四步走\begin{itemize}
                \item 建立TCP连接
                \item 客户端请求内容
                \item 服务器回复
                \item 关闭连接
            \end{itemize}
        \end{itemize}

    \end{frame}

    \begin{frame}
        \frametitle{传输什么样的内容?}
    
        \begin{itemize}
            \item 静态内容\begin{itemize}
                \item html页面等已有的内容->直接发过去
                \item 类比printf("Hello world!");
            \end{itemize}
            \item 动态内容\begin{itemize}
                \item 需要运行程序才能知道发送什么->fork一下并运行
                \item 类比printf("\%ld",time());
            \end{itemize}
            \item web内容与服务器上的文件相联系
        \end{itemize}
    
    \end{frame}

    \begin{frame}
        \frametitle{动态内容怎样传递参数?}
    
        \begin{itemize}
            \item 一切web内容利用URL来定位(类比指针)
            \item 静态内容:\begin{itemize}
                \item 写明http://pkuszm.cn/index.html则发送这个文件
                \item 只写http://pkuszm.cn/则发送默认文件
            \end{itemize}
            \item 动态内容:https://cn.bing.com/search?q=清华大学\begin{itemize}
                \item 通常长成...?...这个样子(后面细讲)
            \end{itemize}
        \end{itemize}
    
    \end{frame}

    \begin{frame}[fragile]
        \frametitle{客户端怎样进行http请求?}
    
        \begin{itemize}
            \item 一个请求行+任意个请求报头+\textbackslash r\textbackslash n
            \item 请求行<method> <uri> <version> \begin{itemize}
                \item method: GET, POST等
                \item uri: 广义的url,类比地址
                \item version: HTTP/1.0或者HTTP/1.1
            \end{itemize}
            \item 请求报头<header name>: <header data>\begin{itemize}
                \item 额外的信息,例如浏览器叫chrome
            \end{itemize}
            \item \textbackslash r\textbackslash n(空行,表示结束)
        \end{itemize}
        \begin{lstlisting}
GET / HTTP/1.1
Host:pkuszm.cn
\r\n
        \end{lstlisting}
    \end{frame}

    \begin{frame}
        \frametitle{服务器怎样进行http响应?}
    
        \begin{itemize}
            \item 一个响应行+任意个响应报头+\textbackslash r\textbackslash n+\textbf{响应主体}
            \item 响应行<version> <status code> <status msg> \begin{itemize}
                \item version: HTTP/1.0或者HTTP/1.1
                \item status code: 200(OK) 301(Moved)等
                \item status msg: 英文描述
            \end{itemize}
            \item 响应报头<header name>: <header data>\begin{itemize}
                \item Content-Type和Content-Length比较重要
            \end{itemize}
            \item \textbackslash r\textbackslash n(空行,表示结束)
            \item 响应主体\begin{itemize}
                \item 分片给出,每片前面有个16进制数表明片的大小(byte为单位,0表示没有)
            \end{itemize}
        \end{itemize}
    \end{frame}

    \begin{frame}[fragile]
        \frametitle{服务器怎样进行http响应?}
        \begin{lstlisting}
HTTP/1.1 301 Moved Permanently
Content-Type: text/html; charset=...
\r\n
15c
<HTML><HEAD>
…
</BODY></HTML>
0
\r\n
        \end{lstlisting}
    \end{frame}
    
    \section{TINY服务器的实现}
    \begin{frame}
        \frametitle{服务器要做什么?}
    
        \begin{itemize}
            \item 与客户端连接
            \item 通过socket读请求
            \item 切割请求行(暂时只支持GET)
            \item 如果uri包含cgi-bin,说明是动态的,fork一个进程去做
            \item 否则是静态的,直接把静态文件发过去
        \end{itemize}
    
    \end{frame}

    \begin{frame}
        \frametitle{如何应对动态请求?}

        \begin{figure}
            \centering
            \subfigure[]{\includegraphics[height=.25\paperheight]{figures/1.png}}
            \subfigure[]{\includegraphics[height=.25\paperheight]{figures/2.png}}
            \subfigure[]{\includegraphics[height=.25\paperheight]{figures/3.png}}
        \end{figure}
    
    \end{frame}

    \begin{frame}
        \frametitle{CGI怎么解决问题?}
    
        \begin{itemize}
            \item Q:客户端怎么将参数传给服务器?
            \item A:在URI中传递\begin{itemize}
                \item \textcolor{gray}{http://add.com/cgi-bin/}\textcolor{blue}{adder}\textcolor{red}{?}\textcolor{dkgreen}{15213}\textcolor{mauve}{\&}\textcolor{dkgreen}{18213}
                \item 运行程序adder,?后接参数,用\&分隔,空格用+或\%20
            \end{itemize}
            \item Q: 服务器如何把参数传给子进程?
            \item A: 通过设置环境变量QUERY\_STRING\begin{itemize}
                \item QUERY\_STRING = “15213\&18213”
                \item adder可以用函数getenv来引用它
            \end{itemize}
            \item Q: 服务器如何把其他信息传给子进程?
            \item A: 通过设置CGI定义的许多环境变量
            \item Q: 服务器如何获得子进程的输出?
            \item A: 子进程输出到stdout,经过服务器用dup2重定向到socket
        \end{itemize}
    
    \end{frame}

    \section{代理}
    \begin{frame}
        \frametitle{什么是代理?}
    
        \begin{figure}
            \centering
            \includegraphics[width=.618\paperwidth]{figures/4.png}

            \includegraphics[width=.618\paperwidth]{figures/5.png}
        \end{figure}
    
    \end{frame}
    
    \section*{致谢}}

    \begin{frame}
        \setbeamertemplate{background}{}
        \frametitle{致谢}
        \begin{columns}
            \begin{column}{.5\linewidth}
                \begin{center}
                    祝大家圣诞快乐!
                
                    谢谢聆听!
                \end{center}
            \end{column}
            \begin{column}{.5\linewidth}
                \includegraphics[width=.4\paperwidth]{figures/misaka558.png}
            \end{column}
        \end{columns}
    \end{frame}
    
\end{document}